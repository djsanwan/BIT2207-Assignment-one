\documentclass[12pt,letterpaper]{article}
\begin{document}
\begin{titlepage}
\begin{center}
\title{COLLEGE OF COMPUTING AND INFORMATION SCIENCES\\ DEPARTMENT OF COMPUTER SCIENCE\\ SCHOOL OF COMPUTING AND INFORMATICS TECHNOLOGY\\}
\maketitle
\end{center}
\begin{flushleft}
\begin{center}
\title{NAME:	     WANGIRA JOHN\\
STUDENTS NUMBER:	 214017326\\
REGISTRATION NUMBER: 14/U/15832/EVE}
\maketitle
\end{center}
\end{flushleft}
\end{titlepage}
  \begin{center}
  \line(1,0){300}\\
  [0.25in]
  \huge{\bfseries A REPORT ABOUT MOBILE PHONE REPAIRING AND SERVICING}\\
  [2mm]
  \line(1,0){200}\\
  \maketitle
  \end{center}
\section{INTRODUCTION}

A mobile phone is an electronic device used for communicating and messaging. Now days this has become very popular and essential need to the society. These are available in various make and brands in the market. There are more than 4.8 million cellular phones in use throughout the world and out of this millions of phones get damaged either by moisture or by scratches, therefore repairing and servicing of mobile phones are also required to be done. This has got a prospective market. 

\subsection{Technical aspect}
Process of repairing and servicing mobile phones involve two parts; one of hardware repairing and another software repairing. Hardware faults may be detected by  and faulty parts may be replaced. The Software part may be rectified with the help of personal computer and CD drive of the software installed in phones. If the fault is related to software the mobile phone is connected to the computers vide data cable and necessary checking is done.

Re-installation of software is done if required. The hardware such as MIC, speaker, LCD display, IC, charging connector. Battery connector, Sim connector, PCB board are checked and necessary repairing is done. Proper training is required for repairing and servicing of mobile phones.
\subsection{Market potential}
Repairing and servicing of mobile phones have a good market prospect all over the country. This business can be started in a very less investment. Though branded companies have their own service center but the demand for repairing of mobile phones is very high per now therefore more service centers are required to avail these services nearer to the people.
\subsection{Pollution control}
Repairing and servicing of mobile phones does not create any pollution. However, efforts need to be made to keep the unit clean and proper disposal of wastes to be done so that there is minimum pollution in the unit.
\subsection{Advantages}

•	Repairing and servicing phones is a financial advantage to any individual willing to join this business because I for one could be their reference.

•	It helps the society in acquiring adequate services for their gadgets since these services are readily available to them in their localities.

•	To customers, the price of repairing a cell phone is quite cost effective than buying a new brand phone from a market in case its damaged or misused.

•	Data loss. One is ought to transfer all data that's present to their damaged

\subsection{Disadvantages}

•	This work is a bit tiresome and requires prior practical skills about phone repairing and servicing

•	There is a tender of most mechanics tempering with the most delicate parts of the phone and end up damaging them of which the burden has to be relieved to them because customers have a saying that ‘I either take it as I brought it or when it health has improved (repaired).’ This could be so disadvantageous to us mechanics just because of that slit mistake you might have caused to their gadget.

 \subsection{Conclusion}
 
 In a nut shell, phone repair and servicing is a bit hectic, tiresome, needs much brain attention and concentration, requires one to have an idea of what every customer's problem requires and at list try to fix for them. Once equaled with these valuations, you are assured of at list having a smiling account every day because mobile phones are being bought and used in every time of the day and destroyed or damaged. It's a professional work that requires more practice than theory.
 
\end{document}
